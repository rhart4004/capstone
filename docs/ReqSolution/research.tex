\documentclass[12pt]{report}

\usepackage{fullpage}
\usepackage{fancyhdr}
\usepackage{url}
\usepackage{color, colortbl}
\definecolor{green}{rgb}{0,1,0}
\usepackage{tabularx,ragged2e,booktabs,caption}

\newcommand{\BibTeX}{{\sc Bib}\TeX}

%%%%%%%%%%%%%%%%%%%%%%%%%%%%%%%%%%%%%%%%%%%%%
%%%%%%				PAGE NUMBERING					       %%%%%%
%%%%%%%%%%%%%%%%%%%%%%%%%%%%%%%%%%%%%%%%%%%%%

\pagestyle{fancy}

\lhead{}
\chead{}
\rhead{\thepage}
\lfoot{}
\cfoot{}
\rfoot{}
\renewcommand{\headrulewidth}{0 pt}
\renewcommand{\footrulewidth}{0 pt}

%%%%%%%%%%%%%%%%%%%%%%%%%%%%%%%%%%%%%%%%%%%%%
%%%%%%				START OF DOCUMENT				       %%%%%%
%%%%%%%%%%%%%%%%%%%%%%%%%%%%%%%%%%%%%%%%%%%%%

\begin{document}

%%%%%%%%%%%%%%%%%%%%%%%%%%%%%%%%%%%%%%%%%%%%%
%%%%%%				TITLE PAGE						       %%%%%%
%%%%%%%%%%%%%%%%%%%%%%%%%%%%%%%%%%%%%%%%%%%%%

\noindent Team \#4 \\  \\
Project \#28: Electronic Parts Picker \\ \\
Stephen Coombes \\
Ryan Hart \\
David Tyler \\ \\
Capstone Proposal \\ \\
\today \\ \\ \\ \\ \\
\centerline{Requirements and Solution Report}
\newpage

%%%%%%%%%%%%%%%%%%%%%%%%%%%%%%%%%%%%%%%%%%%%%
%%%%%%				END TITLE						       %%%%%%
%%%%%%				START DOCUMENT					       %%%%%%
%%%%%%%%%%%%%%%%%%%%%%%%%%%%%%%%%%%%%%%%%%%%%

\section*{Introduction}

This document describes the objectives of the capstone project being done by Team \#4. Our project is an electronic parts picker for Professor Darish of the University of Massachusetts Lowell. Professor Darish is currently running all of the labs and is looking to change up how the labs are run. Professor Darish is looking to have all of the students receive a box of parts at the beginning of the semester with upwards of a couple hundred parts. With the current situation in the parts room, this would be impossible. We started to research how others have built electronic parts pickers in the past and found that there really isn't anything like this. We found a device that would give a specific number of screws and a lego sorting machine. We also found a couple of "robotic hand" ideas that we will look further into. 


\section*{Client Background}

The client is Professor Michael Darish of University of Massachusetts Lowell`s Electrical and Computer Engineering department.  Professor Darish orchestrates the four essential ECE laboratory courses, encompassing two hundred students or more each semester.  The laboratory experiments that he utilizes in these courses necessitate parts being supplied, and with the number of students involved, it is a time-consuming process to acquire parts sufficient for all students.  Currently, the stockroom organizes parts by dividing them into sections and looking up the row and column of each part in a specific section. This involves having a human cross-reference the section, row, and column number. Prof. Darish has now expressed interest in being able to provide the students with all parts needed for the entire semester during the first class meeting.  This would prove exceedingly time-consuming and troublesome for the current method to fulfill the order, so there is a use available for an automated parts picker.  

\section*{Problem Statement}

The parts room has selected hours of operation and due to the current set up gathering parts for each class and project takes too long and is too labor intensive. Professor Darish is looking to have all of the students receive a box of parts at the beginning of the semester with upwards of a couple hundred parts. With the current situation in the parts room, this would be impossible. Our project will resolve this situation. 

\section*{Project Objectives}

The goal is to create a programmable system that can pick out the correct parts in the desired quantities, and separate them for distribution in desired configuration.  The goal for our project is do be able to pick up 10 different parts with 100\% accuracy. 

\section*{Research Summary}
In brief, no evidence can be found of this sort of a system being implemented anywhere else. However, the parts picking robots have been implemented in similar domains successfully so we will explore a couple of those systems. 
	
	
One of the major concerns for this design is how to pick up loose parts from a drawer, and to do so accurately, i.e. picking up a known quantity without error.  The simplest solution appears to be to pick up a single part at a time, and repeat as many times as necessary to obtain further quantities of a part.  A combination of people from Cornell University, University of Chicago, and iRobot created a robotic gripper hand with a balloon of coffee grounds and a vacuum pump \cite{universalGrabber}.  This gripper was able to deform to any object, and demonstrated the ability to lift 650 grams in a single pull, pour water from a glass, write with a pen, and lift a raw egg.  It was also able to pick up an LED, which would indicate it has the fine control to pick up other electronic parts.  Another way we found this being done was in a screw dispenser robot from Design Tool Inc. Their robot basically vibrates and augers screws and other fasteners from an open bowel into a jig that feeds into a pneumatic delivery system \cite{screwDispenser}. The system seems to have perfect accuracy based on demonstrations and provides timely, bulk dispensing. The drawback seems to be the power required to run it and the noise it generates from the vibrating bowel and compressed air. It also looks like it has to be reconfigured each time the type of part it is dispensing changes.


The closest machine we found to what we want to do was actually a Lego sorting machine made out of Legos. The machine has a vibrating hopper that angles towards a lift that lifts one piece at a time into a channel. The channel then uses a kind of airlock system to separate individual pieces out since they are all in a line\cite{legoSorter}.


We also found a couple sources that discussed how best to pick up electronic parts. One source advocates using a small diameter rubber tube that can be used to pickup parts with suction via a vacuum pump \cite{vacuumPick}. The idea specifically applied to a human-controlled tool but could possibly be extended for use in an automatic system. This doesn't seem to apply to resistors, but looks like it might be the best way to pickup integrated circuits. Another method was simply using tweezers to pickup the parts. Not sure how applicable this is for our project, but it is something to keep in mind.

\section*{Requirements}

% What the Device WILL Do

\noindent The device shall: \\
\begin{itemize}
\item Be able to accurately locate up to 10 different parts
\item Be able to acquire the correct quantity of parts 100\% of the time
\item Be able to place the collected parts into container without dropping any parts
\item Include a user interface which you can specify parts to be collected (no limits accept for container size)
\item Be able to match or exceed a human`s pace (Speed to be determined)
\item Include an emergency shut off that kills the system within 1 second
\item Be able to update an inventory of 10 different parts
\item Reduce human involvement in the parts room by 40\%
\end{itemize}

% What the Device WONT Do

\noindent The device shall not: \\
\begin{itemize}
\item In any way represent a hazard to persons in the room
\item Be able to reload the supplies on its own
\end{itemize}

\section*{Potential Solutions}
Our final project will be the combination of three separate projects that are combined at the end. The first part of our project is the grabbing mechanism. This will be the device that actually grabs the parts from the storage bins and is responsible for the accuracy of our system. The second part of the project will be the storage of the parts. The third aspect of our project is the control of the two above systems and the software that will allow our system to actually function. This currently doesn't have a metric or ideas since the mechanical design of the system significantly impacts this part of the project.
\subsection*{Grabber}
There are several ideas for the grabber device. These include a vacuum grabber, a mechanical grabber, a system similar to the one shown in \cite{legoSorter}, Cornell's balloon hand \cite{universalGrabber}, a conveyer belt system, and one that cuts the parts off of a roll. These systems all have pros and cons which are described below.

\subsubsection*{Vacuum Grabber}
The vacuum grabber would use a very small head which would pick up components one at a time and store them. The grabber would move on a 3 axis system similar to that of a 3-D printer. \\ \\
Pros
\begin{itemize}
\item Can be completely autonomous, assuming there are parts in the bin
\item There are no extra safely concerns because the air pressure would be so small
\end{itemize}
Cons
\begin{itemize}
\item Picks parts one at a time
\item May not be able to grab a single part consistently
\item Pneumatic systems can leak, decreasing the reliability of the system
\end{itemize}

\subsubsection*{Mechanical Grabber}
The mechanical grabber would be a pair of finger that would grab components. It would use an optical check to make sure that there was only one part held. The grabber would move on a 3 axis system similar to that of a 3-D printer. \\ \\
Pros
\begin{itemize}
\item Can be completely autonomous, assuming there are parts in the bin
\item There are no extra safely concerns because the squeezing power of the grabber would be minimal
\end{itemize}
Cons
\begin{itemize}
\item May not be able to grab a single part consistently
\item Could damage part while picking it up
\item Picks parts one at a time
\end{itemize}

\subsubsection*{Lego System}
The mechanical grabber would be a pair of finger that would grab components. It would use an optical check to make sure that there was only one part held. The grabber would move on a 3 axis system similar to that of a 3-D printer. \\ \\
Pros
\begin{itemize}
\item Can be completely autonomous, assuming there are parts in the bin
\item There are no extra safely concerns because the squeezing power of the grabber would be minimal
\end{itemize}
Cons
\begin{itemize}
\item May not be able to grab a single part consistently
\item Could damage part while picking it up
\item Picks parts one at a time
\end{itemize}

\subsubsection*{Balloon Hand}
The mechanical grabber would be a pair of finger that would grab components. It would use an optical check to make sure that there was only one part held. The grabber would move on a 3 axis system similar to that of a 3-D printer. \\ \\
Pros
\begin{itemize}
\item Can be completely autonomous, assuming there are parts in the bin
\item There are no extra safely concerns because the squeezing power of the grabber would be minimal
\end{itemize}
Cons
\begin{itemize}
\item May not be able to grab a single part consistently
\item Could damage part while picking it up
\item Picks parts one at a time
\end{itemize}

\subsubsection*{Pre-Organized Storage}
The mechanical grabber would be a pair of finger that would grab components. It would use an optical check to make sure that there was only one part held. The grabber would move on a 3 axis system similar to that of a 3-D printer. \\ \\
Pros
\begin{itemize}
\item Can be completely autonomous, assuming there are parts in the bin
\item There are no extra safely concerns because the squeezing power of the grabber would be minimal
\end{itemize}
Cons
\begin{itemize}
\item May not be able to grab a single part consistently
\item Could damage part while picking it up
\item Picks parts one at a time
\end{itemize}

\subsubsection*{Roll Cutter}
The mechanical grabber would be a pair of finger that would grab components. It would use an optical check to make sure that there was only one part held. The grabber would move on a 3 axis system similar to that of a 3-D printer. \\ \\
Pros
\begin{itemize}
\item Can be completely autonomous, assuming there are parts in the bin
\item There are no extra safely concerns because the squeezing power of the grabber would be minimal
\end{itemize}
Cons
\begin{itemize}
\item May not be able to grab a single part consistently
\item Could damage part while picking it up
\item Picks parts one at a time
\end{itemize}

\subsection*{Storage}

\subsubsection*{Wall of Bins}
The mechanical grabber would be a pair of finger that would grab components. It would use an optical check to make sure that there was only one part held. The grabber would move on a 3 axis system similar to that of a 3-D printer. \\ \\
Pros
\begin{itemize}
\item Can be completely autonomous, assuming there are parts in the bin
\item There are no extra safely concerns because the squeezing power of the grabber would be minimal
\end{itemize}
Cons
\begin{itemize}
\item May not be able to grab a single part consistently
\item Could damage part while picking it up
\item Picks parts one at a time
\end{itemize}

\subsubsection*{Spooling Printer Style}
The mechanical grabber would be a pair of finger that would grab components. It would use an optical check to make sure that there was only one part held. The grabber would move on a 3 axis system similar to that of a 3-D printer. \\ \\
Pros
\begin{itemize}
\item Can be completely autonomous, assuming there are parts in the bin
\item There are no extra safely concerns because the squeezing power of the grabber would be minimal
\end{itemize}
Cons
\begin{itemize}
\item May not be able to grab a single part consistently
\item Could damage part while picking it up
\item Picks parts one at a time
\end{itemize}

\subsubsection*{Spinning Columns of Bins}
The mechanical grabber would be a pair of finger that would grab components. It would use an optical check to make sure that there was only one part held. The grabber would move on a 3 axis system similar to that of a 3-D printer. \\ \\
Pros
\begin{itemize}
\item Can be completely autonomous, assuming there are parts in the bin
\item There are no extra safely concerns because the squeezing power of the grabber would be minimal
\end{itemize}
Cons
\begin{itemize}
\item May not be able to grab a single part consistently
\item Could damage part while picking it up
\item Picks parts one at a time
\end{itemize}

\subsubsection*{Table of Bins}
The mechanical grabber would be a pair of finger that would grab components. It would use an optical check to make sure that there was only one part held. The grabber would move on a 3 axis system similar to that of a 3-D printer. \\ \\
Pros
\begin{itemize}
\item Can be completely autonomous, assuming there are parts in the bin
\item There are no extra safely concerns because the squeezing power of the grabber would be minimal
\end{itemize}
Cons
\begin{itemize}
\item May not be able to grab a single part consistently
\item Could damage part while picking it up
\item Picks parts one at a time
\end{itemize}

\section*{Chosen Solution}
\subsection*{Grabber}

% Start Metric for Grabber Mechanism

\begin{minipage}{\linewidth}
\centering
\captionof{table}{Grabber Mechanism Metric} \label{tab:grabber} 
\begin{tabular}{|l|l|l|l|l|l|l|l|}
\hline
 & Accuracy & Safety & Speed & Cost & Autonomy & Reliability & Total \\
 \hline
Vacuum & 2 & 4 & 2 & 2 & 4 & 3 & 2.7 \\
Mechanical Grabber & 4 & 4 & 2 & 2 & 4 & 3.5 & 3.35 \\
\rowcolor{green}
Lego System & 4.5 & 4 & 4 & 3 & 4 & 3.5 & 3.875 \\
Balloon Hand & 1 & 5 & 1 & 5 & 4 & 3 & 2.65 \\
Pre-Organized Storage & 4.5 & 4 & 5 & 0 & 2 & 4 & 3.575 \\
Roll Cutter & 5 & 4 & 4 & 4 & 4 & 2 & 3.65 \\
Weights & 0.25 & 0.05 & 0.15 & 0.1 & 0.15 & 0.3 & 1 \\
\hline
\end{tabular}\par
\bigskip
Should be a caption
\end{minipage}

% End Metric for Grabber Mechanism

\subsection*{Storage}

% Start Metric for Storage Mechanism

\begin{minipage}{\linewidth}
\centering
\captionof{table}{Storage Mechanism Metric} \label{tab:grabber} 
\begin{tabular}{|l|l|l|l|l|l|}
\hline
 & Size & Cost & Scalability & Reliability & Total \\
 \hline
 \rowcolor{green}
Wall of Bins & 4 & 4 & 4 & 5 & 4.2 \\
Spooling Printer Style & 4 & 1 & 4 & 2 & 3.3 \\
Spinning Column of Bins & 2 & 2 & 1 & 4 & 2 \\
Table of Bins & 1 & 4 & 3 & 5 & 2.9 \\
Weights & 0.3 & 0.1 & 0.4 & 0.2 & 1 \\
\hline
\end{tabular}\par
\bigskip
Should be a caption
\end{minipage}

% End Metric for Storage Mechanism


\section*{Hazard Assessment}

The main hazard is that with the grabber mechanism that was chosen, there will be the system that moves it around to each bin which could potentially hurt someone in the room. We plan to mitigate this hazard by having an emergency stop button mounted near the door so that at anytime a human can stop our device.  

\section*{Conclusion}

This document has presented the overall progress for our electronic parts picker for the client, Professor Darish, of Massachusetts Lowell. Professor Darish was looking for a way to give out parts that would allow him to be able to give each student a box full of all the parts they would need for the semester. We are building an electronic parts picker that will allow him to have this become a reality. From our research we are pioneering the idea of an electronic parts picker since we found almost nothing during our research. We only found devices people had built that do part of what we are attempting, either sorting or dispensing, but nothing that was able to do both and non with electronic components. 

\bibliographystyle{ieeetr}
\bibliography{bib}

\end{document}
