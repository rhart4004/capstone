\documentclass[12pt]{report}

\usepackage{fullpage}
\usepackage{fancyhdr}

\pagestyle{fancy}

\lhead{}
\chead{}
\rhead{\thepage}
\lfoot{}
\cfoot{}
\rfoot{}
\renewcommand{\headrulewidth}{0 pt}
\renewcommand{\footrulewidth}{0 pt}

\begin{document}


\noindent Team \#4 \\  \\
Project \#28: Electronic Parts Picker \\ \\
Stephen Coombes \\
Ryan Hart \\
David Tyler \\ \\
Capstone Proposal \\ \\
\today \\ \\ \\ \\ \\
\centerline{Project Objectives Report}
\newpage

\section*{Introduction}

This document describes the objectives of the capstone project being done by Team \#4. It will provide relevant information on the client, outline the problem, and quantify the objectives of the project.


\section*{Client Background}

The client is Professor Michael Darish of University of Massachusetts Lowell`s Electrical and Computer Engineering department.  Professor Darish orchestrates the four essential ECE laboratory courses, encompassing two hundred students or more each semester.  The laboratory experiments that he utilizes in these courses necessitate parts being supplied, and with the number of students involved, it is a time-consuming process to acquire parts sufficient for all students.  Currently, the stockroom organizes parts by dividing them into sections and looking up the row and column of each part in a specific section. This involves having a human cross-reference the section, row, and column number. Prof. Darish has now expressed interest in being able to provide the students with all parts needed for the entire semester during the first class meeting.  This would prove exceedingly time-consuming and troublesome for the current method to fulfill the order, so there is a use available for an automated parts picker.  

\section*{Problem Statement}

The parts room has selected hours of operation and due to the current set up gathering parts for each class and project takes too long and is too labor intensive.

\section*{Project Objectives}

The goal is to create a programmable system that can pick out the correct parts in the desired quantities, and separate them for distribution in desired configuration.  \\ \\

\noindent The device shall: \\
\begin{itemize}
\item Be able to accurately locate up to 10 different parts
\item Be able to acquire the correct quantity of parts 100\% of the time
\item Be able to place the collected parts into container without dropping any parts
\item Include a user interface which you can specify parts to be collected (no limits accept for container size)
\item Be able to match or exceed a human`s pace (Speed to be determined)
\item Include an emergency shut off that kills the system within 1 second
\item Be able to update an inventory of 10 different parts
\item Reduce human involvement in the parts room by 40\%
\end{itemize}


\noindent The device shall not: \\
\begin{itemize}
\item In any way represent a hazard to persons in the room
\item Be able to reload the supplies on its own
\end{itemize}

\section*{Conclusion}

This document has presented the problem and objectives for an electronic parts picker for the client, Professor Darish of 
 of Massachusetts Lowell.  The next stage will be to consider and select an approach to implementing this parts picker.  


\end{document}
